%%------------------------------------------------------
% 
%% UNIVERSIDADE FEDERAL DE SANTA CATARINA - UFSC
%
%% Prof.: Wyllian B. da Silva
%%
%% Template: estilo IEEEtran [paper com duas colunas]
%% Adaptado de: https://ieeeauthorcenter.ieee.org/create-your-ieee-article/use-authoring-tools-and-ieee-article-templates/ieee-article-templates/templates-for-transactions/
%               https://ctan.org/tex-archive/macros/latex/contrib/IEEEtran?lang=en

%% Instruções: http://mirrors.ctan.org/macros/latex/contrib/IEEEtran/IEEEtran_HOWTO.pdf
%%
%% Recomendações:
%% Utilize o Edito Kile (SO Linux)
%% Certifique-se de que a codificação de caracteres utilizada é a UTF-8





\documentclass[journal]{IEEEtran}


%%------------------------------------------------------
%% Packages
%%------------------------------------------------------
\usepackage[T1]{fontenc}           %% Codificação de caracteres
\usepackage[utf8]{inputenc}        %% Codificação de caracteres (conversão automática dos acentos)
\usepackage[dvips]{graphicx}       %% para a macro includegraphics 
\usepackage[english,brazil]{babel} %% PT_BR e EN (o último define a prioridade no arquivo)
\usepackage{pgf}                   %% macro para criar gráficos
\usepackage{epsfig}                %% or use the epsfig package if you prefer to use the old commands
\usepackage{graphics}              %% use the graphics package for simple commands
\usepackage{graphicx}              %% or use the graphicx package for more complicated commands
\usepackage{epstopdf}              %% enable EPS (convert to PDF)
\usepackage{float}                 %% float environment
\usepackage{eqparbox}              %% to define a group of boxes 
\usepackage{hyphenat}              %% prevent hyphenation
\usepackage{hyperref}              %% enalbe one-click link

% \usepackage{showframe} %% just for the example

% \usepackage[sort,compress]{cite} %% disable if natbib package is activated
\usepackage[numbers,sort&compress,square]{natbib} %% e.g., [2-5]



%%------------------------------------------------------
%% Definitions
%%------------------------------------------------------

\hyphenation{op-tical net-works semi-conduc-tor}

%% Can use something like this to put references on a page
%% by themselves when using endfloat and the captionsoff option.
\ifCLASSOPTIONcaptionsoff
  \newpage
\fi


%%----------------- Definindo as variáveis com números
\makeatletter
%
\newcommand{\prenome}{\afterassignment\prenome@aux\count0=}
\newcommand{\prenome@aux}{\csname prenome\the\count0\endcsname}
%
\newcommand{\nomedomeio}{\afterassignment\nomedomeio@aux\count0=}
\newcommand{\nomedomeio@aux}{\csname nomedomeio\the\count0\endcsname}
%
\newcommand{\sobrenome}{\afterassignment\sobrenome@aux\count0=}
\newcommand{\sobrenome@aux}{\csname sobrenome\the\count0\endcsname}
\makeatother
%%%%%

%%----------------- Configurações de hyperlinks
%% Não decorado, sem destaque
\hypersetup{
  colorlinks=false,
  pdfborder={0 0 0},
}




%%------------------------------------------------------
%% Configurações
%%------------------------------------------------------

%%----------------- Título
\title                                                {Spin Coater de Baixo Custo Desenvolvido com Motor sem Escovas DC}

\newcommand{\emailautor}                              {matheus.w.g@grad.ufsc.br}

\newcommand{\siglaRevista}                            {UFSC}

\newcommand{\Revista}                                 {Universidade Federal de Santa Catarina (UFSC)}



%%----------------- Autor Principal (a acenturação deverá ser indireta)
\newcommand{\prenomePrincipal}                        {Matheus}
\newcommand{\nomedomeioPrincipal}                     {Wilgen}
\newcommand{\sobrenomePrincipal}                      {Gonçalves}



%%------------------------------------------------------
%% Autor(es)


%%----------------- Apenas um autor
\author{\IEEEauthorblockN{\prenomePrincipal~\nomedomeioPrincipal~\sobrenomePrincipal\IEEEauthorrefmark{1} - 17150183}

\IEEEauthorblockA{\IEEEauthorrefmark{1}Universidade Federal de Santa Catarina (UFSC)}% <-this % stops an unwanted space

\IEEEauthorblockA{\IEEEauthorrefmark{1}Projeto Integrador - EMB5636}% <-this % stops an unwanted space

\IEEEauthorblockA{\IEEEauthorrefmark{1}Professor Wyllian Bezerra da Silva}% <-this % stops an unwanted space
%%

\thanks{\Revista. Correspond\^encia ao autor: \prenomePrincipal~\nomedomeioPrincipal~\sobrenomePrincipal~(email: \emailautor).}}



%%------------------------------------------------------
%% Cabeçalho
\markboth{\MakeUppercase{\Revista}}%% acentuação indireta
%% Apenas um autor:
{\sobrenomePrincipal: \MakeUppercase{\Revista}}


%%------------------------------------------------------
%% Abstract
\IEEEtitleabstractindextext{

  {\selectlanguage{brazil}
    \begin{abstract}
    O procedimento de spin coating é simples e eficaz na tarefa de formar um filme
fino uniforme sobre superfícies planas, é possível alcançar espessuras da ordem de
micrometros a nanômetros, este fato faz das máquinas de spin coating muito desejadas
em laboratórios de filmes finos e tratamentos de superfícies. Este trabalho terá o
objetivo de produzir uma máquina de spin coating com baixo custo em relação as
comerciais, também tem como objetivo o desenvolvimento de um driver para motores
sem escovas DC de HD’s, será utilizado processos de fabricação de PCI, programação
em C++ e microcontroladores Atmel. Spin coating é amplamente utilizado para o estudo de filmes finos uniformes, em aplicações que incluem sensores, sistemas microeletromecânicos e circuitos integrados.

    \end{abstract}
    %%----------------- Keywords
    \renewcommand\IEEEkeywordsname{Palavras-chave} %% Palavras-chave ao invés de 'Index Terms'
    \begin{IEEEkeywords}
    Spin Coater, Arduino e Disco Rígido.
    \end{IEEEkeywords}
  }
  
}



\begin{document}



%%------------------------------------------------------
%% Inserção de informações
\maketitle
\IEEEdisplaynontitleabstractindextext
\IEEEpeerreviewmaketitle


%%------------------------------------------------------
%% Section
\section{R1 - Descrição}

\IEEEPARstart{O} processo de Spin Coating consiste em fixar um substrato através de vácuo ou material adesivo, com o centro de massa centralizado com o rotor da máquina utilizada, adiciona-se a solução desejada sobre o substrato para que seja produzido um filme fino, neste processo a rotação retira o excesso de solução, por meio da força centrípeta. Para esta finalidade é utilizado uma máquina que tem a capacidade de controlar a rotação do substrato, este controle pode ser feito através da manipulação direta do usuário ou de um modo de operação automático da ferramenta.


\begin{table}[!htbp]
\renewcommand{\arraystretch}{1.3}
\caption{Cronograma de atividades}
\label{tab:tabela1}
\centering
\begin{tabular}{c|l|l|l|l|l|l|l|l|l|}
\cline{2-10}
\multicolumn{1}{l|}{}                                                                                                         & \multicolumn{9}{c|}{Semana}                                                                                                                                                                                                    \\ \hline
\multicolumn{1}{|c|}{Atividade}                                                                                               & \multicolumn{1}{c|}{1} & \multicolumn{1}{c|}{2} & \multicolumn{1}{c|}{3} & \multicolumn{1}{c|}{4} & \multicolumn{1}{c|}{5} & \multicolumn{1}{c|}{6} & \multicolumn{1}{c|}{7} & \multicolumn{1}{c|}{8} & \multicolumn{1}{c|}{9} \\ \hline
\multicolumn{1}{|c|}{1 - Escrever relatório 1}                                                                                & \multicolumn{1}{c|}{X} &                        &                        &                        &                        &                        &                        &                        &                        \\ \hline
\multicolumn{1}{|c|}{2 - Levantar a relação de materiais}                                                                     & \multicolumn{1}{c|}{X} & \multicolumn{1}{c|}{}  & \multicolumn{1}{c|}{}  & \multicolumn{1}{c|}{}  & \multicolumn{1}{c|}{}  & \multicolumn{1}{c|}{}  & \multicolumn{1}{c|}{}  & \multicolumn{1}{c|}{}  &                        \\ \hline
\multicolumn{1}{|c|}{\begin{tabular}[c]{@{}c@{}}3 - Estudar e pesquisar o assunto \\ e trabalhos relacionados\end{tabular}}   & X                      & \multicolumn{1}{c|}{X} &                        &                        &                        &                        &                        &                        &                        \\ \hline
\multicolumn{1}{|c|}{4 - Escrever relatório 2}                                                                                &                        & \multicolumn{1}{c|}{X} &                        &                        &                        &                        &                        &                        &                        \\ \hline
\multicolumn{1}{|c|}{5 - Controlar o motor de HD}                                                                             & \multicolumn{1}{c|}{}  & \multicolumn{1}{c|}{}  & \multicolumn{1}{c|}{}  & \multicolumn{1}{c|}{}  & \multicolumn{1}{c|}{}  & \multicolumn{1}{c|}{}  & \multicolumn{1}{c|}{}  & \multicolumn{1}{c|}{}  &                        \\ \hline
\multicolumn{1}{|c|}{6 - Escrever relatório 3}                                                                                &                        &                        &                        &                        &                        &                        &                        &                        &                        \\ \hline
\multicolumn{1}{|c|}{7 - Programar o Arduino (Display)}                                                                       & \multicolumn{1}{c|}{}  & \multicolumn{1}{c|}{}  & \multicolumn{1}{c|}{}  & \multicolumn{1}{c|}{}  & \multicolumn{1}{c|}{}  & \multicolumn{1}{c|}{}  & \multicolumn{1}{c|}{}  & \multicolumn{1}{c|}{}  &                        \\ \hline
\multicolumn{1}{|c|}{\begin{tabular}[c]{@{}c@{}}8 - Fazer modos de controle \\ onde o Arduino comanda o\\ motor\end{tabular}} & \multicolumn{1}{c|}{}  & \multicolumn{1}{c|}{}  & \multicolumn{1}{c|}{}  & \multicolumn{1}{c|}{}  & \multicolumn{1}{c|}{}  & \multicolumn{1}{c|}{}  & \multicolumn{1}{c|}{}  & \multicolumn{1}{c|}{}  &                        \\ \hline
\multicolumn{1}{|c|}{9 - Testes preliminares}                                                                                 &                        &                        &                        &                        &                        &                        &                        &                        &                        \\ \hline
\multicolumn{1}{|c|}{10 - Escrever relatório 4}                                                                               &                        &                        &                        &                        &                        &                        &                        &                        &                        \\ \hline
\multicolumn{1}{|c|}{11 - Realizar a montagem da máquina}                                                                     &                        &                        &                        &                        &                        &                        &                        &                        &                        \\ \hline
\multicolumn{1}{|c|}{12 - Realizar Testes}                                                                                    & \multicolumn{1}{c|}{}  & \multicolumn{1}{c|}{}  & \multicolumn{1}{c|}{}  & \multicolumn{1}{c|}{}  & \multicolumn{1}{c|}{}  & \multicolumn{1}{c|}{}  & \multicolumn{1}{c|}{}  & \multicolumn{1}{c|}{}  &                        \\ \hline
\multicolumn{1}{|c|}{13 - Escrever relatório 5}                                                                               &                        &                        &                        &                        &                        &                        &                        &                        &                        \\ \hline
\multicolumn{1}{|c|}{14 - Correção de Erros na máquina}                                                                       & \multicolumn{1}{c|}{}  & \multicolumn{1}{c|}{}  & \multicolumn{1}{c|}{}  & \multicolumn{1}{c|}{}  & \multicolumn{1}{c|}{}  & \multicolumn{1}{c|}{}  & \multicolumn{1}{c|}{}  & \multicolumn{1}{c|}{}  &                        \\ \hline
\multicolumn{1}{|c|}{15 - Experimentos com a máquina}                                                                         &                        &                        &                        &                        &                        &                        &                        &                        &                        \\ \hline
\multicolumn{1}{|c|}{16 - Escrever relatório 6}                                                                               &                        &                        &                        &                        &                        &                        &                        &                        &                        \\ \hline
\multicolumn{1}{|c|}{17 - Revisão da redação}                                                                                 &                        &                        &                        &                        &                        &                        &                        &                        &                        \\ \hline
\multicolumn{1}{|c|}{18 - Escrever relatório 7}                                                                               &                        &                        &                        &                        &                        &                        &                        &                        &                        \\ \hline
\multicolumn{1}{|c|}{19 - Entrega do Trabalho}                                                                                & \multicolumn{1}{c|}{}  & \multicolumn{1}{c|}{}  & \multicolumn{1}{c|}{}  & \multicolumn{1}{c|}{}  & \multicolumn{1}{c|}{}  & \multicolumn{1}{c|}{}  & \multicolumn{1}{c|}{}  & \multicolumn{1}{c|}{}  &                        \\ \hline
\end{tabular}
\end{table}



%%------------------------------------------------------
%% Section
\section{R2 - Relação dos materiais e custos}
Foram feitas alterações no relatório 1, anteriormente foi escrito no programa Microsoft Word, agora está sendo empregado o uso do \LaTeX. A descrição anterior foi modificada para descrever melhor a máquina, foi inserido maiores informações ao cronograma mostrado na \cite{tab:tabela2}, desta maneira pode-se verificar de forma mais clara as atividades a serem implementadas.

Os materiais que serão utilizados podem ser visualizados de forma abrangente na, entre os componentes está o display que servirá como o meio para a comunicação entre o usuário e a máquina por meio do arduino, o motor de disco rígido foi uma gentileza de uma loja que faz manutenção de computadores.Os sensores Hall 3144 serão para extrair dados de rotação e em conjunto com outros componentes eletrônicos realizarará a rotação do motor. As chaves táctil, acabamentos dos botões e potenciômetro serão para realizar a entrada de dados pelo usuário. Os demais componentes eletrônicos como exemplo dos diodos, capacitores, resistores, amplificador operacional, regulador de tensão, transistores e a placa cobreada tem o objetivo de realizar a placa controladora do motor.


\begin{table}[!htpb]
\renewcommand{\arraystretch}{1.3}
\caption{Relação dos Materiais e Custos}
\label{tab:tabela2}
\centering
\begin{tabular}{c|c|c|}
\hline
\multicolumn{1}{|l|}{\textbf{Quantidade}} & \multicolumn{1}{l|}{\textbf{Nome do Componente}} & \multicolumn{1}{l|}{\textbf{Valor {[}R\${]}}} \\ \hline
\multicolumn{1}{|c|}{2x}                  & Chave táctil                                     & 0,50                                          \\ \hline
\multicolumn{1}{|c|}{1x}                  & Motor HD                                         & 0,00                                          \\ \hline
\multicolumn{1}{|c|}{2x}                  & Acabamentos dos Botões                           & 1,00                                          \\ \hline
\multicolumn{1}{|c|}{1x}                  & Arduino UNO                                      & 54,90                                         \\ \hline
\multicolumn{1}{|c|}{1x}                  & Potenciômetro 10k                                & 1,50                                          \\ \hline
\multicolumn{1}{|c|}{1x}                  & Fonte 12V                                        & 15,00                                         \\ \hline
\multicolumn{1}{|c|}{3x}                  & Sensor Hall 3144                                 & 9,00                                          \\ \hline
\multicolumn{1}{|c|}{3x}                  & Transistor Tip 126                               & 5,70                                          \\ \hline
\multicolumn{1}{|c|}{1x}                  & Amplificador Operacional Lm 741                  & 1,60                                          \\ \hline
\multicolumn{1}{|c|}{1x}                  & Regulador de Tensão Lm 350                       & 7,46                                          \\ \hline
\multicolumn{1}{|c|}{3x}                  & Diodo 1N5337                                     & 1,80                                          \\ \hline
\multicolumn{1}{|c|}{2x}                  & Diodo 1n5404                                     & 0,70                                          \\ \hline
\multicolumn{1}{|c|}{4x}                  & Diodo 1n4007                                     & 0,40                                          \\ \hline
\multicolumn{1}{|c|}{1x}                  & Caixa para intalações Elétricas                  & 50,00                                         \\ \hline
\multicolumn{1}{|c|}{1x}                  & Capacitores e Resistores                         & 2,00                                          \\ \hline
\multicolumn{1}{|c|}{1x}                  & Display 16x2 Azul                                & 16,90                                         \\ \hline
\multicolumn{1}{|c|}{1x}                  & Knob para Potenciômetro                          & 5,70                                          \\ \hline
\multicolumn{1}{|c|}{1x}                  & Placa cobreada de fibra                          & 11,80                                         \\ \hline
\multicolumn{1}{l|}{}                     & \multicolumn{1}{r|}{\textbf{Total}}              & 185,96                                        \\ \cline{2-3} 
\end{tabular}
\end{table}





%%------------------------------------------------------
%% Section (no numbering, use section* for acknowledgment)
%% UFSC: no necessary
% \section*{Acknowledgment}
% The authors would like to thank...



%%------------------------------------------------------
%% References (Option 1): extern file
%% Edit with JabRef, for instance
%%\bibliographystyle{IEEEtran} %% Estilo da referência
%%\bibliography{references}    %% Caminho do arquivo (sem extensão)



% %%------------------------------------------------------
% %% References (Option 2): incorporeted
% 
% \begin{thebibliography}{10} 
% 
%   \bibitem{Kopka:1999}
%     H.~Kopka and P.~W. Daly, \emph{A Guide to
%     \LaTeX}, 3rd~ed. Harlow, 
%     England: Addison-Wesley, 1999.
% 
%   \bibitem{Huynen:1998} 
%     Huynen, M.~A. and Bork, P. 1998. Measuring 
%     genome evolution. {\em Proceedings of the 
%     National Academy of Sciences USA} 
%     95:5849--5856.
% 
% 
% \end{thebibliography}



\end{document}
