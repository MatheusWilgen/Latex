%%------------------------------------------------------
% 
%% UNIVERSIDADE FEDERAL DE SANTA CATARINA - UFSC
%
%% Prof.: Wyllian B. da Silva
%%
%% Template: estilo IEEEtran [paper com duas colunas]
%% Adaptado de: https://ieeeauthorcenter.ieee.org/create-your-ieee-article/use-authoring-tools-and-ieee-article-templates/ieee-article-templates/templates-for-transactions/
%               https://ctan.org/tex-archive/macros/latex/contrib/IEEEtran?lang=en

%% Instruções: http://mirrors.ctan.org/macros/latex/contrib/IEEEtran/IEEEtran_HOWTO.pdf
%%
%% Recomendações:
%% Utilize o Edito Kile (SO Linux)
%% Certifique-se de que a codificação de caracteres utilizada é a UTF-8





\documentclass[journal]{IEEEtran}


%%------------------------------------------------------
%% Packages
%%------------------------------------------------------
\usepackage[T1]{fontenc}           %% Codificação de caracteres
\usepackage[utf8]{inputenc}        %% Codificação de caracteres (conversão automática dos acentos)
\usepackage[dvips]{graphicx}       %% para a macro includegraphics 
\usepackage[english,brazil]{babel} %% PT_BR e EN (o último define a prioridade no arquivo)
\usepackage{pgf}                   %% macro para criar gráficos
\usepackage{epsfig}                %% or use the epsfig package if you prefer to use the old commands
\usepackage{graphics}              %% use the graphics package for simple commands
\usepackage{graphicx}              %% or use the graphicx package for more complicated commands
\usepackage{epstopdf}              %% enable EPS (convert to PDF)
\usepackage{float}                 %% float environment
\usepackage{eqparbox}              %% to define a group of boxes 
\usepackage{hyphenat}              %% prevent hyphenation
\usepackage{hyperref}              %% enalbe one-click link

% \usepackage{showframe} %% just for the example

%%% Pacote extra: (Ajuda do Professor para centralizar a tabela 1)
\usepackage{tabularx}
\usepackage{multirow} 
\usepackage{multicol}
\usepackage{lipsum}
\usepackage{soul}
\usepackage{color}

% \usepackage[sort,compress]{cite} %% disable if natbib package is activated
\usepackage[numbers,sort&compress,square]{natbib} %% e.g., [2-5]



%%------------------------------------------------------
%% Definitions
%%------------------------------------------------------

\hyphenation{op-tical net-works semi-conduc-tor}

%% Can use something like this to put references on a page
%% by themselves when using endfloat and the captionsoff option.
\ifCLASSOPTIONcaptionsoff
  \newpage
\fi


%%----------------- Definindo as variáveis com números
\makeatletter
%
\newcommand{\prenome}{\afterassignment\prenome@aux\count0=}
\newcommand{\prenome@aux}{\csname prenome\the\count0\endcsname}
%
\newcommand{\nomedomeio}{\afterassignment\nomedomeio@aux\count0=}
\newcommand{\nomedomeio@aux}{\csname nomedomeio\the\count0\endcsname}
%
\newcommand{\sobrenome}{\afterassignment\sobrenome@aux\count0=}
\newcommand{\sobrenome@aux}{\csname sobrenome\the\count0\endcsname}
\makeatother
%%%%%

%%----------------- Configurações de hyperlinks
%% Não decorado, sem destaque
\hypersetup{
  colorlinks=false,
  pdfborder={0 0 0},
}

%Definiçoes que o Professor fez para centralizar a tabela 1
%Definitions on tabularx:\newcolumntype{R}{>{\raggedleft\arraybackslash}X}
\newcolumntype{L}{>{\raggedright\arraybackslash}X}
\newcolumntype{C}{>{\centering\arraybackslash}X}
\newcolumntype{J}{>{\justifying\arraybackslash}X}


%%------------------------------------------------------
%% Configurações
%%------------------------------------------------------

%%----------------- Título
\title                                                {Spin Coater de Baixo Custo Desenvolvido com Motor sem Escovas DC}

\newcommand{\emailautor}                              {matheus.w.g@grad.ufsc.br}

\newcommand{\siglaRevista}                            {UFSC}

\newcommand{\Revista}                                 {Universidade Federal de Santa Catarina (UFSC)}



%%----------------- Autor Principal (a acenturação deverá ser indireta)
\newcommand{\prenomePrincipal}                        {Matheus}
\newcommand{\nomedomeioPrincipal}                     {Wilgen}
\newcommand{\sobrenomePrincipal}                      {Gonçalves}



%%------------------------------------------------------
%% Autor(es)


%%----------------- Apenas um autor
\author{\IEEEauthorblockN{\prenomePrincipal~\nomedomeioPrincipal~\sobrenomePrincipal\IEEEauthorrefmark{1} - 17150183}

\IEEEauthorblockA{\IEEEauthorrefmark{1}Universidade Federal de Santa Catarina (UFSC)}% <-this % stops an unwanted space

\IEEEauthorblockA{\IEEEauthorrefmark{1}Projeto Integrador - EMB5636}% <-this % stops an unwanted space

\IEEEauthorblockA{\IEEEauthorrefmark{1}Professor Wyllian Bezerra da Silva}% <-this % stops an unwanted space
%%

\thanks{\Revista. Correspond\^encia ao autor: \prenomePrincipal~\nomedomeioPrincipal~\sobrenomePrincipal~(email: \emailautor).}}



%%------------------------------------------------------
%% Cabeçalho
\markboth{\MakeUppercase{\Revista}}%% acentuação indireta
%% Apenas um autor:
{\sobrenomePrincipal: \MakeUppercase{\Revista}}


%%------------------------------------------------------
%% Abstract
\IEEEtitleabstractindextext{

  {\selectlanguage{brazil}
    \begin{abstract}
    O procedimento de spin coating é simples e eficaz na tarefa de formar um filme
fino uniforme sobre superfícies planas, é possível alcançar espessuras da ordem de
micrometros a nanômetros, este fato faz das máquinas de spin coating muito desejadas
em laboratórios de filmes finos e tratamentos de superfícies. Este trabalho terá o
objetivo de produzir uma máquina de spin coating com baixo custo em relação as
comerciais, também tem como objetivo o desenvolvimento de um \textit{driver} para motores
sem escovas DC de HD’s, será utilizado processos de fabricação de PCI, programação
na linguagem C++ e microcontroladores Atmel. Spin coating é amplamente utilizado para o estudo de filmes finos uniformes, 
em aplicações que incluem sensores, sistemas microeletromecânicos e circuitos integrados.

    \end{abstract}
    %%----------------- Keywords
    \renewcommand\IEEEkeywordsname{Palavras-chave} %% Palavras-chave ao invés de 'Index Terms'
    \begin{IEEEkeywords}
    Spin Coater, Arduino e Disco Rígido.
    \end{IEEEkeywords}
  }
  
}



\begin{document}



%%------------------------------------------------------
%% Inserção de informações
\maketitle
\IEEEdisplaynontitleabstractindextext
\IEEEpeerreviewmaketitle


%%------------------------------------------------------
%% Section
\section{R1 - Descrição}

\IEEEPARstart{O} processo de Spin Coating consiste em fixar um substrato através de vácuo ou material adesivo, com o centro de massa centralizado com o rotor da máquina utilizada, adiciona-se a solução desejada sobre o substrato para que seja produzido um filme fino, neste processo a rotação retira o excesso de solução, por meio da força centrípeta. Para esta finalidade é utilizado uma máquina que tem a capacidade de controlar a rotação do substrato, este controle pode ser feito através da manipulação direta do usuário ou de um modo de operação automático da ferramenta.

\begin{table}[!htbp]
{%
\newcommand{\mc}[3]{\multicolumn{#1}{#2}{#3}}
\newcommand{\mr}[3]{\multirow{#1}{#2}{#3}}
\caption{Cronograma de atividades}
\label{tab:tabela1}
\begin{center}
\begin{tabularx}{\columnwidth}{|L|c|cccccccc}\hline
\mc{1}{|c|}{\mr{2}{*}{Atividade}} & \mc{9}{c|}{Semana}\\ \cline{2-10}
 & 1 & \mc{1}{c|}{2} & \mc{1}{c|}{3} & \mc{1}{c|}{4} & \mc{1}{c|}{5} & \mc{1}{c|}{6} & \mc{1}{c|}{7} & \mc{1}{c|}{8} & \mc{1}{c|}{9}\\\hline
Elaboração do $R_1$ & X & \mc{1}{c|}{} & \mc{1}{c|}{} & \mc{1}{c|}{} & \mc{1}{c|}{} & \mc{1}{c|}{} & \mc{1}{c|}{} & \mc{1}{c|}{} & \mc{1}{c|}{}\\\hline
Levantamento dos materiais & X & \mc{1}{c|}{} & \mc{1}{c|}{} & \mc{1}{c|}{} & \mc{1}{c|}{} & \mc{1}{c|}{} & \mc{1}{c|}{} & \mc{1}{c|}{} & \mc{1}{c|}{}\\\hline
Estudo e pequisa sobre o assunto com trabalhos relacionados & X & \mc{1}{c|}{X} & \mc{1}{c|}{} & \mc{1}{c|}{} & \mc{1}{c|}{} & \mc{1}{c|}{} & \mc{1}{c|}{} & \mc{1}{c|}{} & \mc{1}{c|}{}\\\hline
Elaboração do $R_2$ & × & \mc{1}{c|}{X} & \mc{1}{c|}{×} & \mc{1}{c|}{×} & \mc{1}{c|}{×} & \mc{1}{c|}{×} & \mc{1}{c|}{×} & \mc{1}{c|}{×} & \mc{1}{c|}{×}\\\hline
Controle do motor de HD &  & \mc{1}{c|}{×} & \mc{1}{c|}{X} & \mc{1}{c|}{×} & \mc{1}{c|}{×} & \mc{1}{c|}{×} & \mc{1}{c|}{×} & \mc{1}{c|}{×} & \mc{1}{c|}{×}\\\hline
Elaboração do $R_3$ & × & \mc{1}{c|}{×} & \mc{1}{c|}{X} & \mc{1}{c|}{×} & \mc{1}{c|}{×} & \mc{1}{c|}{×} & \mc{1}{c|}{×} & \mc{1}{c|}{×} & \mc{1}{c|}{×}\\\hline
Programação da Interface entre Arduino e Display & × & \mc{1}{c|}{×} & \mc{1}{c|}{X} & \mc{1}{c|}{×} & \mc{1}{c|}{×} & \mc{1}{c|}{×} & \mc{1}{c|}{×} & \mc{1}{c|}{×} & \mc{1}{c|}{×}\\\hline
Elaborar modos de controle entre o Arduino e o Motor & × & \mc{1}{c|}{×} & \mc{1}{c|}{×} & \mc{1}{c|}{×} & \mc{1}{c|}{×} & \mc{1}{c|}{×} & \mc{1}{c|}{×} & \mc{1}{c|}{×} & \mc{1}{c|}{×}\\\hline
Testes preliminares & × & \mc{1}{c|}{×} & \mc{1}{c|}{×} & \mc{1}{c|}{×} & \mc{1}{c|}{×} & \mc{1}{c|}{×} & \mc{1}{c|}{×} & \mc{1}{c|}{×} & \mc{1}{c|}{×}\\\hline
Escrever $R_4$ & × & \mc{1}{c|}{×} & \mc{1}{c|}{×} & \mc{1}{c|}{×} & \mc{1}{c|}{×} & \mc{1}{c|}{×} & \mc{1}{c|}{×} & \mc{1}{c|}{×} & \mc{1}{c|}{×}\\\hline
Montagem da máquina & × & \mc{1}{c|}{×} & \mc{1}{c|}{×} & \mc{1}{c|}{×} & \mc{1}{c|}{×} & \mc{1}{c|}{×} & \mc{1}{c|}{×} & \mc{1}{c|}{×} & \mc{1}{c|}{×}\\\hline
Testes Definitivos & × & \mc{1}{c|}{×} & \mc{1}{c|}{×} & \mc{1}{c|}{×} & \mc{1}{c|}{×} & \mc{1}{c|}{×} & \mc{1}{c|}{×} & \mc{1}{c|}{×} & \mc{1}{c|}{×}\\\hline
Elaboração do $R_5$ & × & \mc{1}{c|}{×} & \mc{1}{c|}{×} & \mc{1}{c|}{×} & \mc{1}{c|}{×} & \mc{1}{c|}{×} & \mc{1}{c|}{×} & \mc{1}{c|}{×} & \mc{1}{c|}{×}\\\hline
Correção de Erros & × & \mc{1}{c|}{×} & \mc{1}{c|}{×} & \mc{1}{c|}{×} & \mc{1}{c|}{×} & \mc{1}{c|}{×} & \mc{1}{c|}{×} & \mc{1}{c|}{×} & \mc{1}{c|}{×}\\\hline
Experimentos & × & \mc{1}{c|}{×} & \mc{1}{c|}{×} & \mc{1}{c|}{×} & \mc{1}{c|}{×} & \mc{1}{c|}{×} & \mc{1}{c|}{×} & \mc{1}{c|}{×} & \mc{1}{c|}{×}\\\hline
Elaboração do $R_6$ & × & \mc{1}{c|}{×} & \mc{1}{c|}{×} & \mc{1}{c|}{×} & \mc{1}{c|}{×} & \mc{1}{c|}{×} & \mc{1}{c|}{×} & \mc{1}{c|}{×} & \mc{1}{c|}{×}\\\hline
Revisão da redação & × & \mc{1}{c|}{×} & \mc{1}{c|}{×} & \mc{1}{c|}{×} & \mc{1}{c|}{×} & \mc{1}{c|}{×} & \mc{1}{c|}{×} & \mc{1}{c|}{×} & \mc{1}{c|}{×}\\\hline
Elaboração do $R_7$ & × & \mc{1}{c|}{×} & \mc{1}{c|}{×} & \mc{1}{c|}{×} & \mc{1}{c|}{×} & \mc{1}{c|}{×} & \mc{1}{c|}{×} & \mc{1}{c|}{×} & \mc{1}{c|}{×}\\\hline
Entrega do Trabalho & × & \mc{1}{c|}{×} & \mc{1}{c|}{×} & \mc{1}{c|}{×} & \mc{1}{c|}{×} & \mc{1}{c|}{×} & \mc{1}{c|}{×} & \mc{1}{c|}{×} & \mc{1}{c|}{×}\\\hline
\end{tabularx}
\end{center}
}%
\end{table}




%%------------------------------------------------------
%% Section
\section{R2 - Relação dos materiais e custos}
Foram feitas alterações no relatório 1, anteriormente foi escrito no programa Microsoft Word, agora está sendo empregado o uso do \LaTeX. A descrição anterior foi modificada para descrever melhor a máquina, foi inserido mais informações ao cronograma mostrado na Tabela~\ref{tab:tabela1}, desta maneira pode-se verificar de forma mais clara as atividades a serem implementadas.

Os materiais que serão utilizados podem ser visualizados de forma abrangente na Tabela~\ref{tab:tabela2}, entre os componentes está o display que servirá como o meio para a comunicação entre o usuário e a máquina por meio do arduino. Os sensores Hall 3144 \cite{Halldata:2005} serão utilizados para extrair dados de rotação e em conjunto com outros componentes eletrônicos realizarar a rotação do motor. As chaves tácteis, acabamentos dos botões e potenciômetro serão para realizar a entrada de dados pelo usuário. Os demais componentes eletrônicos como exemplo dos diodos, capacitores, resistores, amplificador operacional, regulador de tensão, transistores e a placa cobreada tem o objetivo de realizar a placa controladora do motor.


\begin{table}[!htpb]
\renewcommand{\arraystretch}{1.3}
\caption{Relação dos Materiais e Custos}
\label{tab:tabela2}
\centering
\begin{tabular}{c|c|c|}
\hline
\multicolumn{1}{|c|}{\textbf{Quantidade}} & \multicolumn{1}{c|}{\textbf{Nome do Componente}} & \multicolumn{1}{c|}{\textbf{Valor {[}R\${]}}} \\ \hline
\multicolumn{1}{|c|}{2x}                  & Chave táctil                                     & 0,50                                          \\ \hline
\multicolumn{1}{|c|}{1x}                  & Motor HD                                         & 0,00                                          \\ \hline
\multicolumn{1}{|c|}{2x}                  & Acabamentos dos Botões                           & 1,00                                          \\ \hline
\multicolumn{1}{|c|}{1x}                  & Arduino UNO                                      & 54,90                                         \\ \hline
\multicolumn{1}{|c|}{1x}                  & Potenciômetro 10k                                & 1,50                                          \\ \hline
\multicolumn{1}{|c|}{1x}                  & Fonte 12V                                        & 15,00                                         \\ \hline
\multicolumn{1}{|c|}{3x}                  & Sensor Hall 3144                                 & 9,00                                          \\ \hline
\multicolumn{1}{|c|}{3x}                  & Transistor Tip 126                               & 5,70                                          \\ \hline
\multicolumn{1}{|c|}{1x}                  & Amplificador Operacional Lm 741                  & 1,60                                          \\ \hline
\multicolumn{1}{|c|}{1x}                  & Regulador de Tensão Lm 350                       & 7,46                                          \\ \hline
\multicolumn{1}{|c|}{3x}                  & Diodo 1N5337                                     & 1,80                                          \\ \hline
\multicolumn{1}{|c|}{2x}                  & Diodo 1n5404                                     & 0,70                                          \\ \hline
\multicolumn{1}{|c|}{4x}                  & Diodo 1n4007                                     & 0,40                                          \\ \hline
\multicolumn{1}{|c|}{1x}                  & Caixa para intalações Elétricas                  & 50,00                                         \\ \hline
\multicolumn{1}{|c|}{1x}                  & Capacitores e Resistores                         & 2,00                                          \\ \hline
\multicolumn{1}{|c|}{1x}                  & Display 16x2 Azul                                & 16,90                                         \\ \hline
\multicolumn{1}{|c|}{1x}                  & Knob para Potenciômetro                          & 5,70                                          \\ \hline
\multicolumn{1}{|c|}{1x}                  & Placa cobreada de fibra                          & 11,80                                         \\ \hline
\multicolumn{1}{l|}{}                     & \multicolumn{1}{r|}{\textbf{Total}}              & 185,96                                        \\ \cline{2-3} 
\end{tabular}
\end{table}



\section{R3 - Introdução}
O procedimento de spin coating é simples, entretanto máquinas comerciais que realizam este procedimento tem um elevado custo,
como exemplo destaco um modelo considerado de baixo custo chamado SCK-300 \cite{SCK-300} tem um valor da ordem de 2 mil reais, este 
modelo é também considerado de construção caseira, outro exemplo, mas relacionado a um modelo profissional cito
a Ossila \cite{Ossila}, esta máquina tem um valor elevado da ordem de 13 mil reais. Diante do exposto surge a ideia de projetar um
spin coater a partir de materiais simples e que alcance bons resultados e por um preço baixo, um trabalho interessante sobre o assunto é o \textit{Spin coater based on brushless dc motor of hard disk drivers}~\cite{SpinCoater:2006}, neste artigo é discutido a possibilidade de construir um \textit{driver} para controlar um motor dc sem escovas, o esquema utilizado é baseado na força eletromotriz gerada nos enrolamentos do motor para obter a posição do rotor, e dessa maneira realizar a comutação nas fases do motor, para gerenciar os sinais é utilizado um microcontrolador PIC, a partir da leitura foi iniciado o desenvolvimento de um \textit{driver} baseado em sensores hall, desta maneira pode-se obter a posição do motor a partir do imã permanente do rotor e assim comutar as fases e extrair informações da velocidade de rotação com o auxílio do microcontrolador Arduino \cite{Arduino}.

\section{Materiais e métodos}
Para a realização do projeto é utilizado o ambiente de desenvolvimento Arduino IDE para programar o Arduino UNO \cite{Arduino}, o Proteus (versão 8) \cite{Proteus} é o software usado para o projeto eletrônico do driver e para realizar as simulações do funcionamento. O motor utilizado foi retirado de um disco rígido de computador a construção deste motor é robusta e visa evitar vibrações nos discos de memória e podem alcançar velocidades altas da ordem de 10000 rpm, estas características são desejadas para a produção de uma máquina de spin coating. Para a interface com o usuário foi utilizado um display 16x2, um potenciometro linear de 10 kΩ, duas chaves tácteis, para alimentar o projeto foi utilizado uma fonte de 12 Volts e 3 Ampéres. 

Tendo em mãos o motor de disco rígido, surge a questão sobre o método de controle, primeiramente optou-se pelo uso de um controlador de velocidade comercial para motores sem escovas, em testes este controlador revelou os problemas na utilização do controle por indução do rotor nas bobinas do motor, ao inserir alguma carga no eixo do rotor o motor, este não era capaz de iniciar o movimento de forma autônoma.
Para ter uma rotação precisa, e continuar com as vantagens do motor de disco rígido, foi
decidido instalar sensores de efeito hall entre as bobinas do motor (figura não inserida), após instalar os
sensores foi desenvolvido o controlador para que fosse possível alterar o comportamento
da rotação e obter autonomia na hora de obter sinais do circuito ou aprimorar a máquina.
Após testar algumas topologias de circuitos, obteve-se o melhor resultado com o controlador
mostrado de forma esquemática na (figura não inserida), cada fase do motor tem a sua posição
representada pelos símbolos F1, F2 e F3 já o terra do motor é ligado ao terra da placa, este controlador
foi projetado para suportar até 3 Ampères, mas em média a 10000 rpm drena da fonte por volta de
1.5 Ampères. Foi desenvolvido também uma placa auxiliar para a interface com o usuário, o
esquemático desta placa pode ser visualizado na (figura não inserida).




\begin{figure}[!htbp]
\centering
\includegraphics[width=.23\columnwidth]{Sensores instalados no motor.jpg}
\caption{Sensores hall instalados no motor.}
\label{fig:sensores_motor}
\end{figure}




%%------------------------------------------------------
%% Section (no numbering, use section* for acknowledgment)
%% UFSC: no necessary
% \section*{Acknowledgment}
% The authors would like to thank...



%%------------------------------------------------------
%% References (Option 1): extern file
%% Edit with JabRef, for instance
\bibliographystyle{IEEEtran} %% Estilo da referência
\bibliography{references}    %% Caminho do arquivo (sem extensão)



% %%------------------------------------------------------
% %% References (Option 2): incorporeted
% 
% \begin{thebibliography}{10} 
% 
%   \bibitem{Kopka:1999}
%     H.~Kopka and P.~W. Daly, \emph{A Guide to
%     \LaTeX}, 3rd~ed. Harlow, 
%     England: Addison-Wesley, 1999.
% 
%   \bibitem{Huynen:1998} 
%     Huynen, M.~A. and Bork, P. 1998. Measuring 
%     genome evolution. {\em Proceedings of the 
%     National Academy of Sciences USA} 
%     95:5849--5856.
% 
% 
% \end{thebibliography}



\end{document}
